\Introduction


В настоящий момент все наблюдаемые формы коммуникаций сводятся к схемам человек-человек  человек-устройство. Но интернет вещей (Internet of Things, IoT) предлагает нам колоссальное интернет-будущее, в котором появятся коммуникации типа машина-машина (Machine-to-Machine, M2M). Это дает возможность для объединения всех коммуникаций в общую инфраструктуру, позволяя не только управлять всем, что находится вокруг нас, но и предоставляя информацию о состоянии этих вещей. \cite{Dovgal}

Стоит отметить, что M2M --- это подмножество IoT, которое явно имеет дело с соединениями между устройствами. IoT и M2M обеспечивают удаленный доступ для обмена информацией между устройствами без участия человека. Ключевое различие между IoT и M2M заключается в том, что M2M — это подключение двух или более устройств к Интернету для обмена данными, а IoT подключает любое устройство к интернету для изменения процессов в окружающем мире при помощи аналитики. M2M --- это то, что обеспечивает интернет вещей связью, без которой IoT был бы невозможен. Идеология интернета вещей направлена на повышение эффективности экономики за счет автоматизации процессов в различных сферах деятельности и исключения из них человека.

% Коммуникации M2M используются для соединения машин с машинами в одной сотовой сети. IoT используется для подключения машин к машинам в разных сотовых сетях.

% В то время как IoT обычно представляет собой любое устройство, которое подключается к Интернету для повышения производительности, M2M обычно представляет собой подключение двух или более устройств к Интернету для обмена данными и аналитики.

% M2M – это то, что обеспечивает интернет вещей» связью, без которой IoT был бы невозможен. Т.е. IoT – это M2M плюс интеллект.



% Традиционная модель интернета вещей заключается в том, что устройство локально собирает данные, а затем отправляет их в облачный сервис. Там эти данные обрабатываются с помощью искусственного интеллекта или машинного обучения для получения бизнес-аналитики. 

Главная задача интернета вещей \cite{Markeeva} --- создание среды, в которой вещи имеют способность слушаться управления, а данные о них могут быть получены и обработаны для выполнения желаемой задачи посредством обучения устройств. Эта концепция позволяет не только объединять предметы материального мира посредством интернета для обмена информацией между ними, но и развивать возможности по накоплению, структурированию и анализу различной информации о поведении людей в городском пространстве, дома и на работе.

В основе большинства решений в области интернета вещей лежит устройство, которое подключается к облачным сервисам для обмена данными. Обработка данных может выполняться либо на самом устройстве, либо на удалённом сервере. Возможности для разработки таких решений обеспечивает операционная система. При выборе подходящей операционной системы для устройства интернета вещей разработчик должен убедиться, что она совместима с необходимым оборудованием, приложениями и требованиями к подключению. Большое разнообразие операционных систем позволяет выбрать подходящий вариант в зависимости от особенностей конкретной задачи.

Целью данной работы является изучение существующих операционных систем для устройств интернета вещей. Для достижения поставленной цели необходимо выполнить следующие задачи:

\begin{enumerate}
	\item[1)] провести анализ предметной области интернета вещей;
	\item[2)] рассмотреть существующие операционные системы для устройств интернета вещей;
	\item[3)] сформулировать критерии сравнения и оценки рассмотренных операционных систем;
	\item[4)] провести сравнительный анализ существующих решений по выделенным критериям. 
\end{enumerate}

%\begin{itemize}
%\item проанализировать существующую всячину;
%\item спроектировать свою, новую всячину;
%\item изготовить всякую всячину;
%\item проверить её работоспособность.
%\end{itemize}
