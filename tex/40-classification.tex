\chapter{Классификация существующих решений}
\label{cha:classification}

В данном разделе будут описаны критерии сравнения операционных систем для устройств интернета вещей

\section{Критерии сравнения операционных систем}

Описание (и желательно обоснование) критериев сравнения.

Критерии:

\begin{itemize}
	\item Тип ядра: монолитное или микроядро;
	\item Тип лицензии;
	\item Поддержка POSIX;
	\item Тип многозадачности: вытесняющая, кооперативная, отсутствует;
	\item Кроссплатформенность или переносимость (бинарный признак);
	\item ????? Открытый/закрытый исходный код;
	\item ????? Поддержка SMP;
	\item ????? Популярность (востребованность) на рынке (мб взять статистику популярности ОС и сравнивать по объёму доли на рынке в \%);
	\item ????? Промышленное применение (бинарный признак, типа может быть только промышленное и частное применение).
\end{itemize}

Наличие сетевых функций не оценивается, так как все ОС, рассматриваемые в данной работе, являются сетевыми.

\section{Сравнительный анализ операционных систем}

Сводная таблица сравнения операционных систем. \newline

Если по всем критериям одна из систем оказывается лучшей из описываемых, то задать себе закономерный вопрос, почему остальные системы ещё используются. Предложить ещё один критерий сравнения.

