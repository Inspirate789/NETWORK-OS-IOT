\Conclusion % заключение к отчёту
% В рамках курсового проекта был реализован программный инструмент для микроконтроллеров семейства STM32, который позволяет  визуализировать трёхмерные модели.

% Были рассмотрены существующие алгоритмы удаления невидимых линий и поверхностей и алгоритмы закраски, проанализированы  их достоинства, недостатки и возможность использования для решения поставленной задачи. С учётом её особенностей были разработаны структуры данных для реализации выбранных алгоритмов.

% Разработанная программа позволяет получать на экране дисплея изображение полигональной модели, заданной пользователем. При разработке были учтены недостатки существующих программных решений для аппаратной платформы STM32.

% В ходе выполнения экспериментальной части работы было установлено, что разработанное программное обеспечение на тестовом оборудовании показывает высокую стабильность. При возрастании нагрузки на систему сохраняются эффективность работы программы и качество получаемого результата. Микроконтроллеры семейства STM32 подходят для решения относительно простых задач компьютерной графики, не требующих построения реалистических изображений с определением проекционных теней, использованием глобальной модели освещения и методов закраски, требующих больших вычислительных затрат.

% Графический инструмент, разработанный в рамках курсового проекта, имеет по меньшей мере два направления дальнейшего развития.
% \begin{enumerate}
	% \item[1.] Разработка новых приложений, использующих разработанный графический инструмент. К этому располагают интерфейс программы и формат задания исходных данных.
	% \item[2.] Перенос разработок с STM32 на отечественные аппаратные платформы. Это возможно благодаря наличию аналогов семейства микроконтроллеров STM32 отечественного производства.
% \end{enumerate}

Эх, вот если бы я был embedded-разработчиком, то в таких-то проектах ставил бы на свои устройства такую ОС, а в других --- другую, потому что потому.

%\begin{figure}
%	За практику я смог рассмотреть и попробовать полный процесс подготовки к разработке законченного программного продукта, а именно составить техническое задание, провести анализ алгоритмов и методов решения поставленной задачи, составить диаграмму классов и примерный каркас программы. Так же были получены первые результаты работы программы- а именно "зашумленные" при помощи дизеринга картинки.
%\end{figure}
%%% Local Variables: 
%%% mode: latex
%%% TeX-master: "rpz"
%%% End: 
