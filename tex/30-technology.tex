\section{Классификация существующих решений}

В данном разделе будут описаны критерии сравнения операционных систем для устройств интернета вещей.



\subsection{Критерии сравнения операционных систем}

Анализ операционных систем для интернета вещей является задачей классификации. Ниже приведены критерии для их сравнения.

\begin{enumerate}[label*=\arabic*.]
	\item \textbf{Тип ядра}. \newline
	Тип ядра ОС определяет её архитектуру и является ключевым фактором, влияющим на производительность и масштабируемость ОС. Рассматриваются следующие типы ядра: монолитное ядро, микроядро, наноядро и гибридное ядро\footnote{Не может быть классифицировано ни как монолитное ядро, ни как микроядро. Гибридное ядро объединяет аспекты этих двух типов.}.
	
	\item \textbf{Тип лицензии}. \newline
	Тип лицензии определяет использование и распространение программного обеспечения, защищённого авторским правом. Лицензия является гарантией того, что издатель ПО, которому принадлежат исключительные права на программу, не подаст в суд на того, кто её использует.
	
	\item \textbf{Поддержка POSIX}. \newline
	Программные интерфейсы приложений, распространённые в традиционных моделях операционных систем, ограничивают переносимость программного кода и могут значительно снизить экономические показатели эффективности \cite{Posix}. Степень POSIX-соответствия операционных систем характеризуют стандартизованные интерфейсы операционных систем и приложений и, следовательно, степень их мобильности на уровне исходного языка. Многие операционные системы, претендующие на POSIX-совместимость, зачастую поддерживают только небольшие подмножества этого стандарта, поэтому рассматриваемый критерий будет представлен как понятие степени: поддержка POSIX может быть полной, частичной или полностью отсутствовать. В таблице сравнения будут использоваться обозначения <<+>>, <<+/->> и <<->> соответственно.
	
	\item \textbf{Тип многозадачности}. \newline
	Многозадачность ОС обеспечивает параллельную (или псевдопараллельную) обработку нескольких задач. По типу многозадачности рассматриваемые операционные системы можно разделить на три категории:
	
	\begin{itemize}[label*=---]
		\item ОС, реализующие вытесняющую многозадачность;
		\item ОС, реализующие кооперативную многозадачность;
		\item ОС, реализующие одновременно оба типа многозадачности.
	\end{itemize}
	
	Кооперативная многозадачность снижает накладные расходы работы системы за счет исключения лишних переключений задач и использования объектов синхронизации. Режим позволяет заранее предопределить последовательность выполнения задач, возлагая, однако, на разработчика ответственность за установку точек передачи управления между задачами.
	
	Вытесняющая многозадачность избавляет разработчика от необходимости планирования задач вручную (планирование с учетом точек и последовательности передачи управления между задачами) и обеспечивает оптимальную загрузку процессора. При вытесняющей многозадачности планирование задач осуществляется на основании их приоритетов, что позволяет разработчику определить критичность каждой задачи в приложении --- аппаратные ресурсы будут распределяться между задачами в зависимости от их критичности.
	
	\item \textbf{Кроссплатформенность}. \newline
	Кроссплатформенность определяет способность операционной системы работать с несколькими аппаратными платформами. Для сравнения ОС по данному критерию будет рассматриваться бинарный признак, определяющий совместимость операционной системы с одной или несколькими архитектурами процессоров. В таблице сравнения будут использоваться обозначения <<->> и <<+>> соответственно.
	
	\item \textbf{Применение}. \newline
	Операционные системы разрабатываются для конкретных целей, определяющих её специфику. По этой причине при выборе ОС для устройств интернета вещей важно понимать, на какие сферы применения она нацелена. Для сравнения операционные системы будут разделены на две категории: для промышленного интернета вещей (Industrial IoT, IIoT) и для интернета вещей в быту (Home IoT, HIoT). Решение о том, к какой категории IoT отнести ту или иную ОС, принималось на основе текущих сфер их применения. Если об этом не имеется достоверных сведений, то на основе характеристик оценивалась возможность масштабирования решений на базе рассматриваемой ОС до промышленных масштабов.
	
\end{enumerate}

Описанные выше характеристики операционных систем, как правило, декларируются в документациях к ним. Также стоит отметить, что при сравнении операционных систем не рассматривается наличие сетевых функций, так как все ОС, рассматриваемые в данной работе, являются сетевыми.



\subsection{Сравнительный анализ операционных систем}

Результаты сравнения операционных систем для устройств интернета вещей приведены в таблице \ref{tbl:cmp}. Для краткости записи в данной таблице используются следующие
обозначения описанных критериев:

\begin{itemize}[label*=---]
	\item К1 --- тип ядра (монолитное, гибридное, микроядро или наноядро);
	\item К2 --- тип лицензии;
	\item К3 --- поддержка POSIX (полная, частичная или полностью отсутствует);
	\item К4 --- тип многозадачности (вытесняющая или кооперативная);
	\item К5 --- кроссплатформенность (поддерживается одна или несколько аппаратных платформ);
	\item К6 --- применение (промышленный интернет вещей или интернет вещей в быту).
\end{itemize}



\clearpage
\noindent
\captionsetup{format=hang,justification=raggedright,
	singlelinecheck=off}
\begin{longtable}[Hc]{|p{2.8cm}|p{2.6cm}|p{2.5cm}|p{0.6cm}|p{3.5cm}|p{0.6cm}|p{1cm}|}
	\caption{Сравнение операционных систем для устройств интернета вещей\label{tbl:cmp}}\\
	\hline
	\multicolumn{1}{|c}{\textbf{ОС}} & \multicolumn{1}{|c|}{\textbf{K1}} &
	\multicolumn{1}{c|}{\textbf{K2}} & \multicolumn{1}{c}{\textbf{K3}} &
	\multicolumn{1}{|c|}{\textbf{К4}} & \multicolumn{1}{c|}{\textbf{K5}} & 
	\multicolumn{1}{c|}{\textbf{K6}}\\
	\hline
	Azure RTOS    		& Наноядро 	 & Microsoft
	\linebreak Software
	\linebreak License     & -   & Вытесняющая 			   & + & IIoT\\*
	\hline
	Azure Sphere  		& Монолитное & GPL-2.0        & +/- & Вытесняющая 			   & + & IIoT\\*
	\hline
	Amazon
	\linebreak FreeRTOS & Микроядро  & MIT                    & +   & Вытесняющая 			   & + & IIoT\\*
	\hline
	Zephyr              & Наноядро 	 & Apache
	\linebreak Licence 2.0 & +/- & Вытесняющая и кооперативная & + & IIoT\\*
	\hline
	ОСРВ МАКС           & Монолитное & BSD\linebreak 3-Clause     & +/- & Вытесняющая и кооперативная & + & IIoT\\
	\hline
	Huawei
	\linebreak LiteOS   & Микроядро  & BSD\linebreak 3-Clause     & +   & Вытесняющая 			   & - & IIoT\\
	\hline
	Windows 10
	\linebreak IoT   	& Гибридное  & Microsoft
	\linebreak Software
	\linebreak License     & -   & Вытесняющая 			   & + & IIoT\\
	\hline
	Contiki-NG          & Монолитное & BSD\linebreak 3-Clause     & +/- & Кооперативная 		   & + & IIoT\\
	\hline
	Mbed~OS       		& Монолитное & Apache
	\linebreak Licence 2.0 & +   & Кооперативная 		   & - & IIoT\\
	\hline
	KasperskyOS         & Микроядро  & Проприе-
	\linebreak тарная 	  & +/- & Не декларирован 		   & + & IIoT\\
	\hline
	TinyOS              & Монолитное & BSD 				      & +   & Кооперативная 		   & + & HIoT\\
	\hline
	Ubuntu~Core         & Монолитное & CC-BY-SA
	\linebreak \mbox{version~3.0}
	\linebreak UK~licence  & +   & Вытесняющая 			   & + & HIoT\\ % Её UI (от старшей Ubuntu) нужен только (индивидуальным) пользователям слабых ПК, но никак не микроконтроллерам
	\hline
	Raspbian            & Монолитное & GNU~GPL 				  & +   & Вытесняющая 			   & - & HIoT\\
	\hline
\end{longtable}



\subsection{Вывод}

% Если по всем критериям одна из систем оказывается лучшей из описываемых, то задать себе закономерный вопрос, почему остальные системы ещё используются. Предложить ещё один критерий сравнения.

% По данным таблицы \ref{tbl:cmp} можно сформулировать рекомендации по применению операционных систем при проектировании приложений для устройств интернета вещей. 

Для промышленного интернета вещей наиболее универсальными, функциональными и масштабируемыми решениями будут Azure Sphere, Windows 10 IoT и Amazon FreeRTOS, которые не уступают конкурентам по рассматриваемым критериям, а также предоставляют интеграцию с многофункциональными облачными сервисами. Стоит отметить, что сервисы Microsoft и Amazon в момент написания данной работы\footnote{Ноябрь 2022 года.} доступны не во всех странах \cite{Microsoft_ban} \cite{Amazon_ban}. По этой причине можно рассмотреть более доступные ОСРВ МАКС и KasperskyOS. Выбор также будет зависеть от того, какой тип ОС больше подходит для решения конкретной задачи: ОС реального времени или ОС разделения времени. 

Для бытового применения наиболее предпочтительными можно считать Ubuntu Core и Raspbian, так как они являются дистрибутивами Linux, основанными на Debian, и предоставляют графический пользовательский интерфейс, подобный тем, которые имеют операционные системы для персональных компьютеров.

\pagebreak