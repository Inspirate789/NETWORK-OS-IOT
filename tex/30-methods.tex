\chapter{Описание существующих решений}
\label{cha:methods}

В данном разделе будет проведён анализ существующих операционных систем для устройств интернета вещей. Рассматриваемые операционные системы будут относиться к одному из двух типов: ОС реального времени и ОС разделения времени.

Стараться указывать тип ядра ОС, а также информацию по остальным критериям сравнения.

\section{Операционные системы реального времени}

\subsection{Azure RTOS}

\cite{OS_questions}



\subsection{(Amazon) FreeRTOS}



\subsection{Zephyr}



\subsection{ОСРВ МАКС}



\section{Операционные системы разделения времени}



\subsection{Windows 10 IoT}

3 версии: Windows 10 IoT Core, Windows 10 IoT Enterprise и Windows IoT Server. \cite{OS_questions}



\subsection{Azure Sphere}

\cite{OS_questions}



\subsection{Azure IoT Edge}

\cite{OS_questions}



\subsection{Azure Sphere}

\cite{OS_questions}



\subsection{Contiki}



\subsection{Mbed OS}



\subsection{Android Things}



\subsection{Huawei LightOS}



\subsection{TinyOS}



\subsection{Ubuntu Core}

Snappy - OS with Ubuntu Core?


\subsection{Raspbian}

Вроде бы сделана только для Raspberry Py, то есть непереносимая.



\subsection{RIOT}



\subsection{Fuchsia OS}



Если этого мало, то здесь (https://www.intuz.com/top-iot-operating-systems-for-iot-devices) есть ещё больше.



%%% Local Variables:

%%% mode: latex
%%% TeX-master: "rpz"
%%% End: