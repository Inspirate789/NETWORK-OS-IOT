\section*{Список использованных источников}
\addcontentsline{toc}{section}{Список использованных источников}

\begingroup
\renewcommand{\section}[2]{}
\begin{thebibliography}{}
	\bibitem{Dovgal}
	Довгаль В. А. Довгаль Д. В. Интернет Вещей: концепция, приложения и задачи // Вестник Адыгейского государственного университета. Серия 4: Естественно-математические и технические науки. —
	2018. — С. 129–135.
	
	\bibitem{Markeeva}
	В. Маркеева А. Интернет вещей (IoT): возможности и угрозы
	для современных организаций // Общество: социология, психология,
	педагогика. — 2016.
	
	\bibitem{Kaspersky}
	Что такое интернет вещей? Определение и описание [Электронный ресурс]. — Режим доступа: \url{https://www.kaspersky.ru/resource-center/definitions/what-is-iot} (дата обращения: 09.11.2022).
	
	\bibitem{Oracle}
	What is IoT? [Электронный ресурс]. — Режим доступа: \url{https://www.oracle.com/internet-of-things/what-is-iot/} (дата обращения:
	09.11.2022).
	
	\bibitem{IoT_overview}
	Рекомендация МСЭ-T Y.2060. Обзор интернета вещей.
	[Электронный ресурс]. — Режим доступа: \url{https://iotas.ru/files/documents/wg/T-REC-Y.2060-201206-I!!PDF-R.pdf} (дата обращения:
	10.11.2022).
	
	\bibitem{Olifer}
	Олифер В. Г. Олифер Н. А. Сетевые операционные системы. —
	Питер, 2009. — ISBN: 9785272001207.
	
	\bibitem{Tanenbaum}
	С. Таненбаум Э. Операционные системы: разработка и реализация. — Питер, 2006. — ISBN: 5-469-00148-2.
	
	\bibitem{Nanokernel}
	Techopedia Explains Nano Kernel [Электронный ресурс]. — Режим доступа: \url{https://www.techopedia.com/definition/27005/nano-kernel} (дата обращения: 26.11.2022).
	
	\bibitem{Cisco}
	Cisco - Networking, Cloud, and Cybersecurity Solutions [Электронный ресурс]. — Режим доступа: \url{https://www.cisco.com/} (дата обращения: 09.11.2022).
	
	\bibitem{Cisco_architecture_img}
	Lamtzidis, Odysseas. (2019). An IoT Edge-as-a-service (Eaas) Distributed Architecture \& Reference Implementation. 10.13140/RG.2.2.32690.76481. 
	
	\bibitem{Cisco_architecture_big}
	Battery draining attacks against edge computing nodes in IoT net¨
	works / Smith Ryan, Palin Daniel, Ioulianou Philokypros, Vassilakis Vassil¨
	ios, and Shahandashti Siamak // Cyber-Physical Systems. — 2020. — 01. —
	Vol. 6. — P. 1–21.
	
	\bibitem{Cisco_architecture_small}
	Masoodhu Banu N. M. Sujatha C. IoT Architecture a Comparative
	Study // International Journal of Pure and Applied Mathematics. — 2017. —
	Vol. 117. — P. 45–49.
	
	\bibitem{OS_questions}
	Руководство по выбору операционной системы для пограничного устройства Интернета вещей [Электронный ресурс]. — Режим доступа: \url{https://www.quarta-embedded.ru/about/statiyi/rukovodstvo-po-vyboru-operacionnoj-sistemy-dlya-pogranichnogo-ustrojstva-interneta-veshchej/} (дата обращения: 10.11.2022).
	
	\bibitem{Azure_RTOS_overview}
	What is Microsoft Azure RTOS? [Электронный ресурс]. — Режим доступа: \url{https://learn.microsoft.com/en-us/azure/rtos/overview-rtos} (дата обращения: 23.11.2022).
	
	\bibitem{Azure_RTOS_picokernel}
	Обзор ОСРВ Azure ThreadX [Электронный ресурс]. — Режим доступа: \url{https://learn.microsoft.com/ru-ru/azure/rtos/threadx/overview-threadx} (дата обращения: 23.11.2022).
	
	\bibitem{Azure_Sphere_main}
	Azure Sphere [Электронный ресурс]. — Режим доступа: \url{https:
	//azure.microsoft.com/en-us/products/azure-sphere/} (дата обращения:
	24.11.2022).
	
	\bibitem{linux}
	Linux [Электронный ресурс]. — Режим доступа: \url{https://www.linux.org/} (дата обращения: 24.11.2022).
	
	\bibitem{Azure_Sphere_terminology}
	Azure Sphere Terminology [Электронный ресурс]. — Режим доступа: \url{https://learn.microsoft.com/en-us/azure-sphere/product-overview/terminology} (дата обращения: 24.11.2022).
	
	\bibitem{Azure_Sphere_overview}
	What is Azure Sphere? [Электронный ресурс]. — Режим доступа: \url{https://learn.microsoft.com/ru-ru/azure-sphere/product-overview/what-is-azure-sphere} (дата обращения: 24.11.2022).
	
	\bibitem{Azure_Sphere_parts}
	The three parts of Azure Sphere [Электронный ресурс]. —
	Режим доступа: \url{https://arstechnica.com/gadgets/2018/04/microsofts-bid-to-secure-the-internet-of-things-custom-linux-custom-chips-azure/}
	(дата обращения: 24.11.2022).
	
	\bibitem{Amazon_FreeRTOS_overview}
	FreeRTOS [Электронный ресурс]. — Режим доступа: \url{https://aws.amazon.com/ru/freertos/} (дата обращения: 24.11.2022).
	
	\bibitem{Amazon_FreeRTOS_features}
	FreeRTOS features [Электронный ресурс]. — Режим доступа: \url{https://aws.amazon.com/ru/freertos/features/} (дата обращения: 24.11.2022).
	
	\bibitem{Amazon_FreeRTOS_core}
	AWS IoT Core [Электронный ресурс]. — Режим доступа: \url{https://aws.amazon.com/ru/iot-core/} (дата обращения: 24.11.2022).
	
	\bibitem{Amazon_FreeRTOS_defender}
	AWS IoT Device Defender [Электронный ресурс]. — Режим доступа: \url{https://aws.amazon.com/ru/iot-device-defender/} (дата обращения: 24.11.2022).
	
	\bibitem{Amazon_FreeRTOS_management}
	AWS IoT Device Management [Электронный ресурс]. — Режим
	доступа: \url{https://aws.amazon.com/ru/iot-device-management/} (дата
	обращения: 24.11.2022).
	
	\bibitem{Embedded_OS_lecture}
	Internet of Things. Embedded Operating Systems [Электронный ресурс]. — Режим доступа: \url{http://ichatz.me/uniroma1/iot-2020/uniroma1-internet_of_things-2020-ichatz-talk3.pdf} (дата обращения:
	24.11.2022).
	
	\bibitem{Zephyr_overview}
	Zephyr introduction [Электронный ресурс]. — Режим доступа:
	\url{https://docs.zephyrproject.org/latest/introduction/index.html} (дата обращения: 24.11.2022).
	
	\bibitem{Zephyr_device_tree}
	Zephyr devicetree [Электронный ресурс]. — Режим доступа:
	\url{https://docs.zephyrproject.org/latest/build/dts/index.html} (дата обращения: 24.11.2022).
	
	\bibitem{MACS_overview}
	МАКС ОСРВ [Электронный ресурс]. — Режим доступа: \url{https://soware.ru/products/macs-rtos} (дата обращения: 24.11.2022).
	
	\bibitem{MACS_Astrosoft}
	Российская операционная система реального времени для IT-оборудования и Интернета вещей [Электронный ресурс]. — Режим
	доступа: \url{https://www.astrosoft.ru/products/development/rtos-macs/}
	(дата обращения: 24.11.2022).
	
	\bibitem{MACS_registry}
	Заявление о включении сведений о программном обеспечении
	в реестр российского программного обеспечения - «Операционная система реального времени для мультиагентных когерентных систем»
	[Электронный ресурс]. — Режим доступа: \url{https://reestr.digital.gov.ru/request/186244} (дата обращения: 24.11.2022).
	
	\bibitem{Milandr}
	Milandr [Электронный ресурс]. — Режим доступа: \url{https://www.milandr.com/} (дата обращения: 24.11.2022).
	
	\bibitem{STMicroelectronics}
	STMicroelectronics: Home [Электронный ресурс]. — Режим доступа: \url{https://www.st.com/content/st_com/en.html} (дата обращения: 24.11.2022).
	
	\bibitem{MACS_POSIX}
	«АстроСофт» представил новую версию ОСРВ МАКС 1.5
	[Электронный ресурс]. — Режим доступа: \url{https://www.astrosoft.ru/about/press_room/news/36002/} (дата обращения: 24.11.2022).
	
	\bibitem{MACS_blog}
	Что такое операционные системы реального времени на
	примере ОСРВ МАКС [Электронный ресурс]. — Режим доступа: \url{https://internetofthings.ru/78-blog/207-chto-takoe-operatsionnye-sistemy-realnogo-vremeni-na-primere-osrv-maks} (дата обращения:
	24.11.2022).
	
	\bibitem{Huawei_LightOS_github}
	Huawei LiteOS source code [Электронный ресурс]. — Режим доступа: \url{https://github.com/LiteOS/LiteOS} (дата обращения:
	24.11.2022).
	
	\bibitem{Huawei_LightOS_about}
	Huawei LiteOS Overview [Электронный ресурс]. — Режим доступа: \url{https://www.huawei.com/minisite/liteos/en/about.html} (дата
	обращения: 24.11.2022).
	
	\bibitem{Huawei_LightOS_main}
	Huawei LiteOS [Электронный ресурс]. — Режим доступа: \url{https://www.huawei.com/minisite/liteos/en/index.html} (дата обращения:
	24.11.2022).
	
	\bibitem{Windows_IoT_overview}
	An overview of Windows 10 IoT [Электронный ресурс]. — Режим доступа: \url{https://learn.microsoft.com/en-us/windows/iot-core/windows-iot} (дата обращения: 27.11.2022).
	
	\bibitem{Azure_sevices}
	Продукты Azure [Электронный ресурс]. — Режим доступа: \url{https://azure.microsoft.com/ru-ru/products/} (дата обращения:
	27.11.2022).
	
	\bibitem{Windows_IoT_Core}
	An overview of Windows 10 IoT Core [Электронный ресурс]. —
	Режим доступа: \url{https://learn.microsoft.com/en-us/windows/iot-core/
	windows-iot-core} (дата обращения: 27.11.2022).
	
	\bibitem{Windows_IoT_Enterprice}
	Getting Started with Windows IoT Enterprise [Электронный ресурс]. — Режим доступа: \url{https://learn.microsoft.com/en-us/windows/iot/iot-enterprise/getting_started} (дата обращения: 27.11.2022).
	
	\bibitem{Windows_Enterprice}
	Windows 10 Enterprise [Электронный ресурс]. — Режим доступа: \url{https://www.microsoft.com/en-us/evalcenter/evaluate-windows-10-enterprise} (дата обращения: 27.11.2022).
	
	\bibitem{Windows_IoT_Server}
	An overview of Windows Server IoT 2019 [Электронный ресурс]. — Режим доступа: \url{https://learn.microsoft.com/en-us/windows/iot-core/windows-server} (дата обращения: 27.11.2022).
	
	\bibitem{Windows_Server}
	Windows Server 2022 | Microsoft [Электронный ресурс]. — Режим доступа: \url{https://www.microsoft.com/en-us/windows-server} (дата
	обращения: 27.11.2022).
	
	\bibitem{Contiki_github}
	Contiki-NG source code [Электронный ресурс]. — Режим доступа: \url{https://github.com/contiki-ng/contiki-ng} (дата обращения:
	25.11.2022).
	
	\bibitem{Contiki_overview}
	Payal Malik Malvika Gupta. An Overview Of Iot
	Operating System: Contiki Os And Its Communication Models //
	INTERNATIONAL JOURNAL OF SCIENTIFIC \& TECHNOLOGY
	RESEARCH. — 2020. — Т. 9.
	
	\bibitem{Mbed_OS_ARM}
	Mbed OS [Электронный ресурс]. — Режим доступа:
	{https://www.arm.com/products/development-tools/embedded-and-software/mbed-os} (дата обращения: 24.11.2022).
	
	\bibitem{Mbed_OS_handbook}
	Homepage - Handbook | Mbed [Электронный ресурс]. — Режим доступа: \url{https://os.mbed.com/handbook/} (дата обращения:
	24.11.2022).
	
	\bibitem{Google_Cloud}
	Google Cloud: Cloud Computing Services [Электронный ресурс]. — Режим доступа: \url{https://cloud.google.com/} (дата обращения:
	24.11.2022).
	
	\bibitem{AWS}
	Cloud Computing Services - Amazon Web Services (AWS) [Электронный ресурс]. — Режим доступа: \url{https://aws.amazon.com/} (дата
	обращения: 24.11.2022).
	
	\bibitem{Azure_cloud}
	Microsoft Azure: Службы облачных вычислений [Электронный
	ресурс]. — Режим доступа: \url{https://azure.microsoft.com/} (дата обращения: 24.11.2022).
	
	\bibitem{Mbed_OS_main}
	Free open source IoT OS and development tools for Arm | Mbed
	[Электронный ресурс]. — Режим доступа: \url{https://os.mbed.com} (дата обращения: 25.11.2022).
	
	\bibitem{KasperskyOS_main}
	KasperskyOS [Электронный ресурс]. — Режим доступа: \url{https://os.kaspersky.ru/} (дата обращения: 29.11.2022).
	
	\bibitem{KasperskyOS_techdata}
	KasperskyOS. Техническая информация. [Электронный ресурс]. — Режим доступа: \url{https://os.kaspersky.ru/wp-content/uploads/sites/14/2018/03/KasperskyOS-TechData-ru.pdf} (дата обращения:
	29.11.2022).
	
	\bibitem{KasperskyOS_1_1_overview}
	Обзор KasperskyOS Community Edition 1.1. Общие сведения.
	[Электронный ресурс]. — Режим доступа: \url{https://support.kaspersky.com/help/KCE/1.1/ru-RU/overview_general_inf.htm} (дата обращения: 29.11.2022).
	
	\bibitem{IPC_book}
	Gray John Shapley. Interprocess Communications in Linux: The
	Nooks \& Crannies. — Pearson, 2003. — ISBN: 0130460427.
	
	\bibitem{TinyOS_main}
	TinyOS Home Page [Электронный ресурс]. — Режим доступа:
	{http://www.tinyos.net/} (дата обращения: 26.11.2022).
	
	\bibitem{TinyOS_BMSTU}
	TinyOS - Национальная библиотека им. Н. Э. Баумана [Электронный ресурс]. — Режим доступа: \url{https://ru.bmstu.wiki/TinyOS}
	(дата обращения: 26.11.2022).
	
	\bibitem{TinyOS_book}
	Werner Weber Jan M. Rabaey Emile Aarts.
	Ambient Intelligence. — Springer Berlin, Heidelberg, 2005. —
	ISBN: 978-3-540-27139-0. — Режим доступа: \url{https://link.springer.com/book/10.1007/b138670}.
	
	\bibitem{Ubuntu_Core_doc}
	Ubuntu Core documentation [Электронный ресурс]. — Режим
	доступа: {https://ubuntu.com/core/docs} (дата обращения: 24.11.2022).
	
	\bibitem{Ubuntu_Desktop}
	Ubuntu: Enterprise Open Source and Linux [Электронный ресурс]. — Режим доступа: \url{https://ubuntu.com/} (дата обращения:
	24.11.2022).
	
	\bibitem{Ubuntu_Core_def}
	Ubuntu Core definition [Электронный ресурс]. — Режим доступа: \url{https://www.techtarget.com/searchitoperations/definition/Ubuntu-Core} (дата обращения: 24.11.2022).
	
	\bibitem{Raspbian_overview}
	Welcome to Raspbian [Электронный ресурс]. — Режим доступа:
	{https://www.raspbian.org/} (дата обращения: 24.11.2022).
	
	\bibitem{Debian_Wheezy}
	Информация о выпуске Debian 7.0 (wheezy) [Электронный ресурс]. — Режим доступа: \url{https://www.debian.org/releases/wheezy/armhf/release-notes.ru.pdf} (дата обращения: 24.11.2022).
	
	\bibitem{LXDE}
	LXDE [Электронный ресурс]. — Режим доступа: \url{http://www.lxde.org/} (дата обращения: 24.11.2022).
	
	\bibitem{Raspbian_customize}
	Customise your Raspberry Pi desktop [Электронный ресурс]. —
	Режим доступа: \url{https://projects.raspberrypi.org/en/projects/custom-pi-desktop/3} (дата обращения: 24.11.2022).
	
	\bibitem{APT}
	Пакетный менеджер APT [Электронный ресурс]. — Режим доступа: \url{https://help.ubuntu.ru/wiki/apt} (дата обращения: 24.11.2022).
	
	\bibitem{Posix}
	Совместимость с POSIX [Электронный ресурс]. — Режим доступа: \url{https://www.swd.ru/print.php3?pid=386} (дата обращения:
	26.11.2022).
	
	\bibitem{Microsoft_ban}
	Microsoft suspends new sales in Russia [Электронный ресурс]. —
	Режим доступа: \url{https://blogs.microsoft.com/on-the-issues/2022/03/04/microsoft-suspends-russia-sales-ukraine-conflict/?icid=mscom_marcom_TS1_Sales_update} (дата обращения: 27.11.2022).
	
	\bibitem{Amazon_ban}
	How Amazon is assisting in Ukraine [Электронный ресурс]. — Режим доступа: \url{https://www.aboutamazon.com/news/community/amazons-assistance-in-ukraine\#March4} (дата обращения: 27.11.2022).
\end{thebibliography}
\endgroup

\pagebreak