\section{Анализ предметной области}

В данном разделе будут представлены основные определения, а также будет описана архитектура интернета вещей и поставлена проблема выбора операционной системы для устройств интернета вещей.

\subsection{Основные определения}

\textbf{Интернет вещей} \cite{Kaspersky} --- это система взаимосвязанных вычислительных устройств, которые могут собирать и передавать данные по беспроводной сети без участия человека. В более широком смысле интернет вещей можно определить как концепцию, описывающую сеть физических объектов (<<вещей>>), в которые встроены датчики, программное обеспечение и другие технологии для обмена данными с другими устройствами и системами через интернет \cite{Oracle}. Под вещью в данном контексте понимается предмет физического мира (физическая вещь) или информационного мира (виртуальная вещь), который может быть идентифицирован и интегрирован в сети связи \cite{IoT_overview}.

\textbf{Операционная система компьютера} \cite{Olifer} --- комплекс взаимосвязанных программ, который действует как интерфейс между приложениями и пользователями, с одной стороны, и аппаратурой компьютера, с другой стороны.

\textbf{Сетевая операционная система} \cite{Olifer} позволяет пользователю работать со своим компьютером как с автономным и добавляет к этому возможнось доступа к информационным и аппаратным ресурсам других компьютеров сети. В идеальном случае сетевая ОС должна представить пользователю сетевые ресурсы не в виде сети, а в виде ресурсов единой централизованной виртуальной машины. Для такой операционной системы используют специальное название --- \textbf{распределённая ОС}, или \textbf{истинно распределённая ОС}.

Многозадачные ОС подразделяются на три типа в соответствии с использованными при их разработке критериями эффективности:
\begin{itemize}[label*=---]
	\item системы пакетной обработки (OC EC);
	\item системы разделения времени (ОС семейств Linux, Windows, MacOS);
	\item системы реального времени (QNX, FreeRTOS).
\end{itemize}
Современные операционные системы, проектируемые для устройств интернета вещей, являются либо системами реального времени, либо системами разделения времени. \textbf{Операционная система реального времени} \cite{Olifer} --- это система, предназначенная для управления физическими объектами (процессами), которая способна обеспечить предсказуемое время реакции в ответ на изменение состояния управляемого объекта (процесса). \textbf{Система разделения времени} \cite{Olifer} --- такая форма организации вычислительного процесса, при которой сразу несколько пользователей одновременно работают на компьютере, причём каждому из них кажется, что он получил компьютер в полное своё распоряжение. Главной целью и критерием эффективности систем разделения времени является обеспечение удобства и эффективности работы пользователей.

Ввиду большого многообразия устройств, используемых в распределённых вычислительных системах, при выборе ОС для интернета вещей большую роль играет переносимость. ОС называют \textbf{переносимой} (portable) \cite{Olifer}, или \textbf{мобильной}, если её код может быть сравнительно легко перенесён с процессора одного типа на процессор другого типа и с аппаратной платформы одного типа на аппаратную платформу другого типа. Мобильность --- не бинарное состояние, а понятие степени.

Наиболее общим подходом к структуризации операционной системы является разделение всех её модулей на две группы: модули ядра, выполняющие основные функции ОС, и вспомогательные модули ОС. Ядро ОС \cite{Olifer} представляет собой сложный многофункциональный комплекс, имеющий многослойную структуру. В развитии современных операционных систем наблюдается тенденция в сторону переноса кода в верхние слои и удалении при этом всего, что только возможно, из режима ядра, оставляя минимальное \textbf{микроядро} \cite{Tanenbaum}. Обычно это осуществляется перекладыванием выполнения большинства задач операционной системы на средства пользовательских процессов. Такой подход способствует переносимости, расширяемости, повышению надёжности системы и создаёт хорошие предпосылки для поддержки распределённых приложений. Модель с микроядром хорошо подходит для поддержки \textbf{распределённых вычислений} \cite{Olifer}, так как использует механизмы, аналогичные сетевым: взаимодействие клиентов и серверов путём обмена сообщениями. Серверы микроядерной ОС могут работать как на одном, так и на разных компьютерах. В этом случае при получении сообщения от приложения микроядро может обработать его самостоятельно и передать локальному серверу или же переслать по сети микроядру, работающему на другом компьютере. В контексте интернета вещей переход к распределённой обработке становится актуальным ввиду минимальных изменений в работе операционной системы --- просто локальный транспорт заменяется сетевым.

Разработчики ОС для встраиваемых систем продолжили идею минимизации и пришли к концепции наноядра. \textbf{Наноядро} \cite{Nanokernel} --- архитектура ядра операционной системы компьютеров, в рамках которой крайне упрощённое и минималистичное ядро выполняет лишь одну задачу --- обработку аппаратных прерываний, генерируемых устройствами компьютера. После обработки прерываний от аппаратуры наноядро, в свою очередь, посылает информацию о результатах обработки (например, полученные с клавиатуры символы) вышележащему программному обеспечению при помощи того же механизма прерываний. Также часто реализуют минимальную поддержку потоков: создание и переключение. В некотором смысле концепция наноядра близка к концепции HAL (Hardware Abstraction Layer), предоставляя вышележащему ПО удобные механизмы абстракции от конкретных устройств и способов обработки их прерываний. Наноядро предлагает аппаратную абстракцию, но без системных служб. В современных микроядрах также отсутствуют системные службы, поэтому термины микроядро и наноядро стали аналогичными.



% Переход к распределённой обработке требует минимальных изменений в работе операционной системы --- просто локальный транспорт заменяется сетевым. В контексте интернета вещей эта концепция приобретает очень большое значение, находя своё применение в разработке распределённых вычислительных систем.

% В \textit{микроядерных} ОС в привилегированном режиме остаётся работать только очень небольшая часть ОС, называемая \textbf{микроядром}. Все остальные высокоуровневые функции ядра оформляются в виде приложений, работающих в пользовательском режиме. Микроядро защищено от остальных частей ОС и приложений. В состав микроядра обычно входят машинно-зависимые модули, а также модули, выполняющие базовые (но не все) функции ядра по управлению процессами, обработке прерываний, управлению виртуальной памятью, пересылке сообщений и управлению устройствами ввода-вывода, связанные с загрузкой или чтением регистров устройств. \cite{Olifer}


\subsection{Архитектура интернета вещей} 

Определение архитектуры интернета вещей является предметом серьезных дискуссий. 
% Однако все согласны с тем, что для того, чтобы концепция интернета вещей работала, она должна должна состоять из сети, датчиков и коммуникаций. 
Одной из наиболее подробных архитектур является семиуровневая модель, предложенная компанией Cisco \cite{Cisco}. Ниже, в соответствии с рисунком \ref{fig:architecture} \cite{Cisco_architecture_img}, описаны её уровни \cite{Cisco_architecture_big} \cite{Cisco_architecture_small}.

\begin{figure}[h]
	\centering
	\includegraphics[width=\textwidth ]{img/Illustration-of-the-IoT-Reference-Model-by-Cisco-1.png}
	\caption{Архитектура интернета вещей, предложенная Cisco}
	\label{fig:architecture}
\end{figure} 

\begin{enumerate}[label*=\arabic*.]
	\item \textbf{Физические устройства и контроллеры} --- это уровень, содержащий вещи в IoT. Сюда входит широкий спектр конечных устройств, которые могут отправлять или получать информацию (например, датчики и считыватели радиочастотной идентификации (RFID)).
	
	\item \textbf{Соединение} --- это уровень, содержащий все компоненты, способные передавать информацию. Передача может осуществляться между устройствами на первом уровне, между компонентами на этом уровне или между первым и третьим уровнем.
	
	\item \textbf{Граничные (туманные) вычисления} (с английского Edge (Fog) computing) --- это первый уровень, на котором происходит обработка данных. Здесь могут собираться и предварительно обрабатываться значительные объёмы информации до того, как они будут переданы в верхние уровни. Данный уровень также позволяет форматировать и декодировать данные до того, как они будут обработаны. % Это может снизить нагрузку на более высокие уровни. 
	
	\item \textbf{Накопление данных} --- это уровень, на котором данные сохраняются, чтобы приложения могли получить к ним доступ в случае необходимости. Как правило, необходимая обработка информации не может быть выполнена на сетевых скоростях, поэтому вычислительная система нуждается в промежуточном хранилище данных. Сохранённые данные также могут быть преобразованы и рекомбинированы, чтобы быть готовыми к использованию на более высоких уровнях. В результате на этом уровне данные в движении преобразуются в данные в состоянии покоя.
	
	\item \textbf{Абстракция данных} --- это уровень, позволяющий хранить данные более эффективным образом для повышения производительности более высоких уровней. На данном уровне над данными могут выполняться операции нормализации, индексирования, форматирования, проверки, консолидации, а также обеспечивается доступ к нескольким хранилищам данных.
	
	\item \textbf{Приложения} --- это уровень, на котором информация, накопленная ранее, интерпретируется приложениями. Именно на этом уровне располагается бизнес-логика приложений.
	
	\item \textbf{Взаимодействие и процессы} --- это уровень, объединяющий всё вместе. Система бесполезна, если информация, предоставляемая на этом уровне, не является полезной. Данные из интернета вещей должны использоваться для принятия обоснованных решений.
	
\end{enumerate}

Операционные системы для интернета вещей принципиально меняют процесс разработки программного обеспечения для систем с многоуровневой архитектурой, описанной выше, так как позволяют разработчикам абстрагироваться от особенностей аппаратуры конкретных устройств и предоставляют шаблоны для создания приложений с определённой архитектурой.

% Интернет Вещей --- это новая концепция, в которой Интернет эволюционирует от объединения компьютеров и людей к объединению умных вещей.

% Базовой идеей IoT является предоставление возможности автономного обмена полезной информацией между уникально идентифицируемыми объектами реального мира. Эти устройства оснащены новейшими технологиями, такими, как радиочастотная идентификация (RFID) и беспроводные сети датчиков (WSNs), и в дальнейшем получают возможность принимать самостоятельные решения в зависимости от того, какое автоматизированное действие выполняется. 

% Тем не менее, у разных людей и организаций есть свои отличающиеся концепции Интернета Вещей.

% 6-УРОВНЕВАЯ АРХИТЕКТУРА IOT

% Операционные системы устройств интернета вещей (ОСИВ, англ. Internet of Things Operating Systems, IoT OS) позволяют оснащать умные устройства, подключенные к интернету или иной сети связи, основными функциями компьютера с учётом ограниченности вычислительных ресурсов и специфических технических особенностей экосистемы интернета вещей. Такие ОС близки к ранее существовавшему классу операционных систем реального времени (ОСРВ, англ. RTOS)



\subsection{Сферы применения интернета вещей}

Чтобы поставить проблему выбора операционной системы для интернета вещей, необходимо обозначить круг задач, решаемых устройствами. Концепция интернета вещей находит своё применение во множестве различных сфер жизнедеятельности человека: как в быту, так и в промышленности. Далее будут рассмотрены сферы, в которых применяются IoT-системы, и обозначены задачи, которые ставятся перед умными устройствами.

\subsubsection{Интернет вещей в быту}

В быту интернет вещей применяется при проектировании умных домов. Системы интернета вещей способны автоматизировать бытовые процессы и исключить непосредственное участие человека. К таким процессам можно отнести дистанционное управление бытовой техникой, музыкальными системами и системами освещения. Автоматизация может выполняться при помощи сбора и анализа статистики о бытовых процессах. Принципы функционирования IoT, описанные на примере бытовых процессов, можно перенести на бесконечное множество любых других, начиная от уличного освещения и управления светофорами, и до управления огромными предприятиями и городами \cite{Kaspersky}.

\subsubsection{Промышленный интернет вещей}

Промышленный Интернет вещей (Industrial IoT, IIoT) \cite{Oracle} относится к применению технологии Интернета вещей в промышленных условиях. В последнее время в промышленности используется межмашинное взаимодействие (M2M) для обеспечения беспроводной автоматизации и управления. Но с появлением облачных и смежных технологий (таких как аналитика и машинное обучение) отрасли могут достичь нового уровня автоматизации и тем самым создать новые модели доходов и бизнеса. Организации, которые лучше всего подходят для IoT, --- это те, которые могут выиграть от использования умных устройств в своих бизнес-процессах. Ниже приведены некоторые распространенные варианты использования IIoT.

\subsubsubsection{Умные города}

В умных городах используются такие устройства интернета вещей, как датчики и счетчики для сбора и анализа данных. Полученные данные могут использоваться для улучшения инфраструктуры, коммунального обслуживания и других городских сервисов.

\subsubsubsection{Производства}

Производители могут получить конкурентное преимущество, используя мониторинг производственных линий, чтобы обеспечить упреждающее обслуживание оборудования с помощью датчиков, обнаруживающих надвигающийся сбой. С помощью оповещений от датчиков производители могут быстро проверять оборудование и ремонтировать его в случае необходимости. Это позволяет компаниям сократить эксплуатационные расходы, а также увеличить время безотказной работы и повысить эффективность оборудования.

\subsubsubsection{Транспорт и логистика}

Транспортные и логистические системы могут извлекать выгоду из приложений IoT, отслеживая параметры перевозимых грузов. Например, можно отслеживать температуру перевозимых продуктов питания и напитков, цветочной и фармацевтической продукции, чтобы отправлять предупреждения, когда температура поднимается или падает до уровня, угрожающего качеству товаров.

\subsubsubsection{Розничная торговля}

Приложения IoT позволяют розничным компаниям управлять запасами, улучшать качество обслуживания клиентов, оптимизировать цепочку поставок и сокращать эксплуатационные расходы \cite{Oracle}.

\subsubsubsection{Государственный сектор}

Преимущества IoT в государственном секторе и других средах, связанных с услугами, достаточно обширны \cite{Oracle}. Например, государственные коммунальные службы могут использовать приложения на базе интернета вещей для уведомления своих пользователей об отключениях и небольших перебоях в подаче воды и электроэнергии. Приложения IoT могут собирать данные о масштабах сбоев и управлять ресурсами, чтобы помогать коммунальным службам быстрее восстанавливать работу после сбоев.

\subsubsubsection{Здравоохранение}

Интернет вещей является важным аспектом \textbf{телемедицины} \cite{Kaspersky} (для обозначения интернета медицинских вещей иногда используется аббревиатура IoMT). Примеры его применения включают удаленную медицинскую диагностику, цифровую передачу медицинских изображений, видеоконсультации со специалистами и прочее. Приложения IoT также используются в носимых устройствах, которые могут контролировать здоровье человека и условия окружающей среды. Такие приложения не только помогают людям следить за состоянием своего здоровья, но и позволяют врачам удаленно наблюдать за пациентами.

\subsubsubsection{Общая безопасность во всех отраслях}

Помимо отслеживания физических показателей, IoT можно использовать для повышения безопасности труда \cite{Oracle}. Сотрудники опасных предприятий, таких как шахты, месторождения и электростанции, должны знать о возможном наступлении опасной ситуации. Когда они подключены к приложениям на основе датчиков IoT, они могут быть уведомлены об угрозах аварий, чтобы предпринять необходимые действия.



\subsection{Проблема выбора операционной системы для интернета вещей}

Операционные системы, предназначенные для интернета вещей, обладают различной функциональностью, которая определяет преимущества и недостатки каждого конкретного решения. Выбранная операционная система должна полностью соответствовать требованиям и ограничениям, предъявляемым к устройствам. Можно выделить четыре основных аспекта, которые разработчики устройств учитывают при выборе ОС для периферийных устройств IoT \cite{OS_questions}.

% Перед принятием окончательного решения необходимо ответить на четыре основных вопроса.


\begin{enumerate}[label*=\arabic*.]
	\item \textbf{Какой уровень надежности и долгосрочной поддержки необходим?} \newline 
	Под надёжностью в данном случае понимается соответствие операционной системы определенным стандартам и сертификациям. Необходимо, чтобы выбранная операционная система обеспечивала варианты, подходящие для конкретных отраслевых стандартов и требований.
	
	\item \textbf{Какие требования предъявляются к производительности?} \newline 
	Выбранная операционная система должна соответствовать требованиям устройства к вычислительной мощности и производительности в реальном времени. Также при ответе на данный вопрос стоит учитывать другие аспекты, связанные с производительностью и оказывающие влияние на выбор ОС. К таким можно отнести энергопотребление и объём памяти устройства, а также требования к времени отклика системы на внешние события.
	
	\item \textbf{Обеспечивает ли операционная система безопасность устройства?} \newline 
	Безопасность --- один из основных факторов, учитываемых при разработке устройств интернета вещей. Это касается и используемой ОС, так как от неё зависит целостность данных, собираемых устройствами. Если система не обеспечивает необходимый уровень безопасности, то это может привести к порче или краже данных, нарушению запланированных процессов.
	
	\item \textbf{Является ли выбранная операционная система масштабируемой?} \newline 
	В связи с тем, что операционные системы, как и любое программное обеспечение, меняются вместе с требованиями к функциям, которые они должны предоставлять. Разработка IoT-устройства с масштабируемой ОС позволяет в будущем адаптироваться к другим задачам без внесения значительных изменений. Масштабируемые ОС могут обрабатывать дополнительные ресурсы без изменения скорости вывода и охватывать несколько устройств. % и географических регионов.
	
\end{enumerate}

\pagebreak